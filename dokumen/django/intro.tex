\setcounter{page}{1}
\section{Pengenalan}
\subsection{Apa itu Django}
Django adalah sebuah web framework opensource berbasis bahasa pemrograman python. Tidak seperti kebanyakan Web framework
yang menggunakan arsitektur pengembangan M-V-C atau Model View Controller, Django menggunakan model pemgembangan 
M-T-V atau Model Template View. 

\subsection{Sejarah}
Django pertama kali dibuat pada musim gugur 2003 oleh Adrian Holovaty dan Simon Willison ketika mereka berdua bekerja sebagai web programer 
pada Surat Kabar Lawrence Journal-World.
\subsection{Fitur}

\begin{itemize}
    \item Memiliki kepustakaan yang luas; dalam distribusi Python telah disediakan modul-modul 'siap pakai' untuk berbagai keperluan.
    \item Memiliki tata bahasa yang jernih dan mudah dipelajari.
    \item Memiliki aturan layout kode sumber yang memudahkan pengecekan, pembacaan kembali dan penulisan ulang kode sumber.
    \item Berorientasi objek.
    \item Memiliki sistem pengelolaan memori otomatis (garbage collection, seperti java)
    \item Modular, mudah dikembangkan dengan menciptakan modul-modul baru; modul-modul tersebut dapat dibangun dengan bahasa Python maupun C/C++.
    \item Memiliki fasilitas pengumpulan sampah otomatis, seperti halnya pada bahasa pemrograman Java, python memiliki fasilitas pengaturan penggunaan ingatan komputer sehingga para pemrogram tidak perlu melakukan pengaturan ingatan komputer secara langsung.
    \item Memiliki banyak faslitas pendukung sehingga mudah dalam pengoperasiannya.
\end{itemize}

\subsection{Dimana digunakan ?}
Penggunaan python saat ini sudah sangat meluas. Berbagai perusahaan IT terkemuka sudah menggunakannya seperti:
\begin{itemize}
    \item  Google
    \item  Bitbucket
    \item  Youtube
    \item  Instagram
    \item  NASA
    \item  Pintarest
\end{itemize}
\newpage 

