\setcounter{page}{1}
\section{Pengenalan}
\subsection{Apa itu Python}
Python adalah bahasa pemrograman interpretatif multiguna dengan filosofi perancangan yang berfokus pada tingkat keterbacaan kode. Python diklaim sebagai bahasa yang menggabungkan kapabilitas, kemampuan, dengan sintaksis kode yang sangat jelas, dan dilengkapi dengan fungsionalitas pustaka standar yang besar serta komprehensif. Python juga didukung oleh komunitas yang besar

\subsection{Sejarah}
Python dikembangkan oleh Guido van Rossum pada tahun 1990 di Stichting Mathematisch Centrum (CWI), Amsterdam sebagai kelanjutan dari bahasa pemrograman ABC. Versi terakhir yang dikeluarkan CWI adalah 1.2

\subsection{Fitur}

\begin{itemize}
    \item Memiliki kepustakaan yang luas; dalam distribusi Python telah disediakan modul-modul 'siap pakai' untuk berbagai keperluan.
    \item Memiliki tata bahasa yang jernih dan mudah dipelajari.
    \item Memiliki aturan layout kode sumber yang memudahkan pengecekan, pembacaan kembali dan penulisan ulang kode sumber.
    \item Berorientasi objek.
    \item Memiliki sistem pengelolaan memori otomatis (garbage collection, seperti java)
    \item Modular, mudah dikembangkan dengan menciptakan modul-modul baru; modul-modul tersebut dapat dibangun dengan bahasa Python maupun C/C++.
    \item Memiliki fasilitas pengumpulan sampah otomatis, seperti halnya pada bahasa pemrograman Java, python memiliki fasilitas pengaturan penggunaan ingatan komputer sehingga para pemrogram tidak perlu melakukan pengaturan ingatan komputer secara langsung.
    \item Memiliki banyak faslitas pendukung sehingga mudah dalam pengoperasiannya.
\end{itemize}

\subsection{Dimana digunakan ?}
Penggunaan python saat ini sudah sangat meluas. Berbagai perusahaan IT terkemuka sudah menggunakannya seperti:
\begin{itemize}
    \item  Google
    \item  Bitbucket
    \item  Youtube
    \item  Instagram
    \item  NASA
    \item  Pintarest
\end{itemize}
Dan masih banyak lagi sektor-sektor teknologi informasi yang menggunakannya. 
\subsection{Kapan digunakan dan kapan tidak?}
Python adalah sebuah bahasa pemrograman skrip atau biasa disebut juga Scripting Language. Karena kemudahaannya python banyak digunakan diberbagai keperluan seperti:
\begin{itemize}
    \item Pemrograman web, sebagai backend.
    \item Pemrograman Desktop.
    \item Skrip Otomasi, untuk setup server misalnya.
    \item Sarana mengajar.
\end{itemize}

Kapankah kita tidak bisa menggunakan atau tidak optimal ketika menggunakan python? Adalah ketika kita membutuhkan beberapa hal berikut:
\begin{itemize}
  \item Membuat realtime system, seperti sistem navigasi pesawat atau roket.
  \item Membuat Driver dari sistem operasi
  \item Membuat Sistem Operasi
  \item Membuat sistem pada device yang memiliki sumberdaya terbatas.
\end{itemize}

\newpage 

